\documentclass[11pt]{article}
\usepackage[letterpaper, margin=1in]{geometry}
\usepackage{etaremune}
\frenchspacing
\setlength\parindent{0pt}

\begin{document}

\textsc{{\huge Steven M. Foltz, PhD}}

\bigskip

Children's Hospital of Philadelphia \hfill Staff Scientist II, Tan Lab \\
Philadelphia, PA \hfill foltzs@chop.edu \\

\emph{Research specialty:} cancer bioinformatics, machine learning, single-cell multiomics data analysis

\hrulefill

\bigskip

\textsc{{\Large Education and Training}}

\bigskip

\textbf{Postdoctoral Fellow, University of Pennsylvania, Philadelphia, PA} \hfill July 2020 -- May 2023 \\
NIH K12 IRACDA Postdoctoral Fellow (PennPORT) \\
\emph{Advisor:} Casey Greene, PhD \\
\emph{Focus:} Machine learning methods development in pediatric cancer genomics

\bigskip

\textbf{PhD, Washington University in St. Louis, St. Louis, MO} \hfill June 2014 -- March 2020 \\
Division of Biology and Biomedical Sciences (DBBS) \\
Human and Statistical Genetics Program \\
Precision Medicine Pathway \\
\emph{Advisor}: Li Ding, PhD \\
\emph{Thesis}: Multi-omics integration for gene fusion discovery and somatic mutation haplotyping in cancer

\bigskip

\textbf{MS Biostatistics, University of Washington, Seattle, WA} \hfill September 2011 -- August 2013 \\
\emph{Advisors}: Brian Browning, PhD, and Sharon Browning, PhD \\
\emph{Thesis}: Rare variant method for identity-by-descent detection in sequence data, and the time to most recent common ancestor given identity-by-descent segment length

\bigskip

\textbf{BS Mathematics, Presbyterian College, Clinton, SC} \hfill August 2006 -- May 2010 \\
\emph{Advisor}: Doug Daniel, PhD \\
\emph{Thesis}: Tiling 2-deficient Rectangles with Trominoes

\hrulefill

\bigskip

\textsc{{\Large Research Experience}}

\bigskip
\textbf{Children's Hospital of Philadelphia, Philadelphia, PA} \hfill May 2023 -- present \\
\emph{Staff Scientist II}, Laboratory of Kai Tan, PhD \\
\emph{Focus:} Single-cell multiomics tool development and data analysis in pediatric oncology
\begin{itemize}
  \item Analyzed tumor evolution and minimal residual disease using single-cell DNA-seq, ATAC-seq, and RNA-seq in high-risk pediatric B-ALL and sarcoma
  \item Led detection of chimeric antigen receptor (CAR) integration sites using longitudinal single-cell ATAC-seq in pediatric B-ALL patients receiving CAR T-cell therapy
\end{itemize}

\bigskip

\textbf{University of Pennsylvania, Philadelphia, PA} \hfill July 2020 -- May 2023 \\
\emph{Postdoctoral Fellow}, Laboratory of Casey Greene, PhD, in partnership with the Childhood Cancer Data Lab at Alex's Lemonade Stand Foundation \\
\emph{Focus:} Machine learning methods development in pediatric cancer genomics
\begin{itemize}
  \item Evaluated methods for cross-platform normalization of gene expression data for machine learning
  \item Developed single-sample prediction models for cancer subtype prediction using gene expression data from multiple platforms, including single-cell data
\end{itemize}

\bigskip

\textbf{Washington University in St. Louis, St. Louis, MO} \hfill June 2014 -- May 2020 \\
\emph{Graduate Student}, Laboratory of Li Ding, PhD \\
\emph{Thesis}: Multi-omics integration for gene fusion discovery and somatic mutation haplotyping in cancer
\begin{itemize}
	\item Led pan-cancer computational projects utilizing tumor and normal tissue sequencing data to understand patterns of germline and somatic variation
	\item Developed integrative data analysis tool and pipeline development, including variant calling and quality control using bulk DNA and RNA, single-cell RNA, and linked-read DNA sequencing
	\item Discovered gene fusions from The Cancer Genome Atlas (9,624 patients, 33 cancer types) and a large multiple myeloma cohort (742 patients), highlighting clinically-relevant events
	\item Analyzed linked-read whole genome sequencing data for phasing somatic mutations, understanding haplotype structures, and modeling tumor evolution
	\item Organized and delivered donor presentations to establish research funding through the Paula C. and Rodger O. Riney Blood Cancer Research Initiative Fund (\$25 million since 2016)
\end{itemize}

\bigskip

\textbf{Virginia Commonwealth University, Richmond, VA} \hfill August 2013 -- May 2014 \\
\emph{Biostatistician}, Laboratory of Nengliang Yao, PhD
\begin{itemize}
	\item Led statistical aspects of cancer care research as part of a healthcare policy team
	\item Procured, cleaned, and analyzed data from national cancer databases, including SEER
	\item Developed geographically weighted regression and spatial panel data models
\end{itemize}

\bigskip

\textbf{University of Washington, Seattle, WA} \hfill September 2011 -- August 2013 \\
\emph{Graduate Research Assistant}, Laboratory of Brian Browning, PhD, and Sharon Browning, PhD \\
\emph{Thesis}: Rare variant method for identity-by-descent detection in sequence data, and the time to most recent common ancestor given identity-by-descent segment length
\begin{itemize}
	\item Implemented novel algorithms to define identity-by-descent (IBD) between individuals using rare DNA variation and developed statistical models for shared ancestor inference
	\item Worked with data from the 1000 Genomes Project and simulated coalescent model data
	\item Assisted in methods development for missing data imputation and haplotype phasing
\end{itemize}

\bigskip

\textbf{Presbyterian College, Clinton, SC} \hfill Spring 2010 \\
\emph{Undergraduate honors research in mathematics}, advised by Doug Daniel, PhD \\
\emph{Thesis}: Tiling 2-deficient Rectangles with Trominoes
\begin{itemize}
	\item Developed analytical and computational approaches to answer an open question about tiling deficient rectangular fields with trominoes
\end{itemize}

\bigskip

\textbf{Medical University of South Carolina, Charleston, SC} \hfill Summer 2009\\
\emph{Undergraduate Summer Research Fellow}, Laboratory of Jim Zheng, PhD
\begin{itemize}
	\item Created a weighted clustering coefficient model to detect regions of high interconnectivity in yeast gene networks based on synthetic lethal interactions
\end{itemize}

\bigskip

\textbf{Presbyterian College, Clinton, SC} \hfill Fall 2008 -- Spring 2009 \\
\emph{Undergraduate Research Assistant}, Laboratory of James Wanliss, PhD
\begin{itemize}
	\item Analyzed space weather data to refine solar storm prediction models
\end{itemize}

\hrulefill

\bigskip

\textsc{\Large Publications}\\ $^*$ co-first authors

\begin{etaremune}

% PLOS style, add \textbf{} for my name, * for co-firsts, \emph{} for journal
% Items organized by publication date

\item \textbf{Foltz SM}, Li Y, Yao L, Terekhanova NV, Weerasinghe A, Gao Q, et al. Somatic mutation phasing and haplotype extension using linked-reads in multiple myeloma. \emph{bioRxiv}. 2024. p. 2024.08.09.607342. (pre-print)

\item Hawkins AG, Shapiro JA, Spielman SJ, Mejia DS, Prasad DV, Ichihara N, et al. The Single-cell Pediatric Cancer Atlas: Data portal and open-source tools for single-cell transcriptomics of pediatric tumors. \emph{bioRxiv}. 2024. p. 2024.04.19.590243. (pre-print)

\item Liang W-W, Lu RJ-H, Jayasinghe RG, \textbf{Foltz SM}, Porta-Pardo E, Geffen Y, et al. Integrative multi-omic cancer profiling reveals DNA methylation patterns associated with therapeutic vulnerability and cell-of-origin. \emph{Cancer Cell}. 2023;41: 1567–1585.e7.

\item Li Y*, Dou Y*, Da Veiga Leprevost F*, Geffen Y*, Calinawan AP*, Aguet F, et al. Proteogenomic data and resources for pan-cancer analysis. \emph{Cancer Cell}. 2023;41: 1397–1406.

\item Shapiro JA, Gaonkar KS, Spielman SJ, Savonen CL, Bethell CJ, Jin R, et al. OpenPBTA: The Open Pediatric Brain Tumor Atlas. \emph{Cell Genom}. 2023;3: 100340.

\item Yao L*, Wang JT*, Jayasinghe RG*, O’Neal J, Tsai C-F, Rettig MP, et al. Single-Cell Discovery and Multiomic Characterization of Therapeutic Targets in Multiple Myeloma. \emph{Cancer Res}. 2023;83: 1214–1233.

\item \textbf{Foltz SM}, Greene CS, Taroni JN. Cross-platform normalization enables machine learning model training on microarray and RNA-seq data simultaneously. \emph{Commun Biol}. 2023;6: 222.

\item Liu R*, Gao Q*, \textbf{Foltz SM*}, Fowles JS, Yao L, Wang JT, et al. Co-evolution of tumor and immune cells during progression of multiple myeloma. \emph{Nat Commun}. 2021;12: 2559.

\item \textbf{Foltz SM}, Gao Q, Yoon CJ, Sun H, Yao L, Li Y, et al. Evolution and structure of clinically relevant gene fusions in multiple myeloma. \emph{Nat Commun}. 2020;11: 2666.

\item McDermott JE, Arshad OA, Petyuk VA, Fu Y, Gritsenko MA, Clauss TR, et al. Proteogenomic Characterization of Ovarian HGSC Implicates Mitotic Kinases, Replication Stress in Observed Chromosomal Instability. \emph{Cell Reports Medicine}. 2020;1: 100004.

\item Sanchez-Vega F, Mina M, Armenia J, Chatila WK, Luna A, La KC, et al. Oncogenic Signaling Pathways in The Cancer Genome Atlas. \emph{Cell}. 2018;173: 321–337.e10.

\item Gao Q*, Liang WW*, \textbf{Foltz SM*}, Mutharasu G, Jayasinghe RG, Cao S, et al. Driver Fusions and Their Implications in the Development and Treatment of Human Cancers. \emph{Cell Rep}. 2018;23: 227–238.e3.

\item Dang HX, White BS, \textbf{Foltz SM}, Miller CA, Luo J, Fields RC, et al. ClonEvol: Clonal ordering and visualization in cancer sequencing. \emph{Ann Oncol}. 2017;28: 3076–3082.

\item \textbf{Foltz SM}, Liang WW, Xie M, Ding L. MIRMMR: Binary classification of microsatellite instability using methylation and mutations. \emph{Bioinformatics}. 2017;33: 3799–3801.

\item Olfson E, Saccone NL, Johnson EO, Chen LS, Culverhouse R, Doheny K, et al. Rare, low frequency and common coding variants in CHRNA5 and their contribution to nicotine dependence in European and African Americans. \emph{Mol Psychiatry}. 2016;21: 601–607.

\item Ye K, Wang J, Jayasinghe R, Lameijer EW, McMichael JF, Ning J, et al. Systematic discovery of complex insertions and deletions in human cancers. \emph{Nat Med}. 2016;22: 97–104.

\item Yao N, \textbf{Foltz SM}, Odisho AY, Wheeler DC. Geographic analysis of urologist density and prostate cancer mortality in the United States. \emph{PLoS One}. 2015;10: e0131578.

\end{etaremune}

\hrulefill

\bigskip

\textsc{{\Large Teaching Experience}}

\bigskip

\textbf{Rutgers University -- Camden, Camden, NJ}

\bigskip

\emph{Adjunct Faculty}, Cancer Data Science (BIO 393) \hfill Spring 2022 \\
Student profile: 20 undergraduate biology majors \\
Responsibilities: Designed and taught a special topics course around my research interests

\bigskip

\emph{Adjunct Faculty}, Statistics for Biological Research (BIO 283) \hfill Fall 2021 \\
Student profile: 24 undergraduate biology majors \\
Responsibilities: Taught an existing course, with major overhauls to computer workshop materials

\bigskip

\textbf{Washington University in St. Louis, St. Louis, MO}

\bigskip

\emph{Instructor}, Introduction to Python -- Bioinformatics \hfill Winter 2019 \\
Student profile: 6 post-baccalaureate research fellows

\bigskip

\emph{Teaching Assistant}, Genomics (Biol 5488) \hfill Spring 2016 \\
Student profile: 40 first-year biomedical graduate students

\bigskip

\emph{Tutor}, Fundamentals of Biostatistics for Graduate Students (Biol 5075) \hfill Fall 2015, Fall 2016 \\
Student profile: 30 first-year biomedical graduate students

\bigskip

\textbf{Presbyterian College, Clinton, SC}

\bigskip

\emph{Tutor}, Department of Mathematics \hfill January 2007 -- April 2010 \\

\hrulefill

\bigskip

\textsc{\Large Mentoring Experience}\\ $^*$ undergraduate co-authors

\bigskip

\textbf{Undergraduate mentoring at Children's Hospital of Philadelphia}

\begin{itemize}
  \item $^*$Avi Loren (Summer 2023 -- present) \hfill UPenn 2025
\end{itemize}

\bigskip

\textbf{Undergraduate mentoring with PennPORT Pals at Rutgers University -- Camden}

\begin{itemize}
	\item John Crespo (Summer 2021 -- Summer 2022) \hfill RUC 2023
\end{itemize}

\bigskip

\textbf{Undergraduate mentoring at Washington University in St. Louis}

\begin{itemize}
	\item $^*$Moses Schindler (Summer -- Fall 2019) \hfill WashU 2023 \\ Tumor evolution modeling and graph-based methods using single cell RNA-seq
	\item Jessika Baral (Fall 2018) \hfill WashU 2021 \\ Haplotype and variant data analysis from 1000 Genomes
	\item $^*$Guanlan Dong (Fall 2018, Spring 2019) \hfill WashU 2019 \\ Haplotype phase sets and genome assembly summarization
	\item Justin Chen (Summer 2018) \hfill Columbia University 2022 \\ Multiple myeloma phased and unphased variant comparison
	\item Edwin Qiu (Spring 2018, Fall 2018) \hfill WashU 2020 \\ Driver fusion prioritization methods development
\end{itemize}

\bigskip

\textbf{High school mentoring with WashU Young Scientist Program}

\begin{itemize}
	\item Jada Reid (Fall 2014 -- Spring 2018) \hfill CSMB 2018
\end{itemize}

\hrulefill

\bigskip

\textsc{\Large Presentations} \\

\textbf{Oral presentations}

\begin{itemize}
  \item ``Multiomic single-cell tumor evolution models of minimal residual disesae in pediatric B-cell acute lymphoblastic leukemia.'' American Association for Cancer Research Annual Meeting. San Diego, CA. April 2024.
  \item ``Cross-platform normalization enables machine learning in rare diseases.'' Rutgers University – Camden, Department of Biology Seminar Series. April 2022.
	\item ``Multiple Myeloma: Groundbreaking DNA, RNA, and protein technologies at WashU.'' Washington University School of Medicine Paula C. and Rodger O. Riney Blood Cancer Research Initiative Advisory Board Meeting. February 2019.
	\item ``RNA-Seq and fusion discovery pipeline for a large multiple myeloma cohort.'' WashU DBBS Precision Medicine Pathway Retreat. October 2017.
	\item ``RNA and DNA methylation signatures for scoring tumor stemness.'' WashU DBBS Precision Medicine Pathway Retreat. September 2016.
\end{itemize}

\textbf{Poster presentations}

\begin{itemize}
  \item ``Medulloblastoma subtype single sample predictor built on multiple gene expression platforms.'' MidAtlantic Bioinformatics Conference. Philadelphia, PA. October 2022.
  \item ``Cross-platform normalization enables machine learning model training on microarray and RNA-Seq data simultaneously.'' NIH IRACDA Conference. Albuquerque, NM. July 2022.
	\item ``Fusion gene detection across a large cohort of multiple myeloma patients.'' International Myeloma Workshop. Boston, MA. September 2019.
	\item ``Comprehensive Multi-Omics Analysis of Gene Fusions in a Large Multiple Myeloma Cohort.'' American Society of Hematology. San Diego, CA. December 2018.
	\item ``Multiple Myeloma RNA-seq and Fusion Discovery Pipeline.'' WashU DBBS Computational and Systems Biology / Human and Statistical Genetics / Molecular Genetics and Genomics Joint Program Retreat. Potosi, MO. September 2017.
	\item ``Genomic region and sample selection strategy for variant discovery and association analysis.'' Genome Informatics. Cold Spring Harbor, NY. October 2015.
	\item  ``Comprehensive Analysis of Germline Complex Insertions and Deletions.'' WashU DBBS Computational and Systems Biology / Molecular Genetics and Genomics Joint Program Retreat. New Haven, MO. September 2015.
	\item  ``Comprehensive Analysis of Germline Complex Insertions and Deletions.'' WashU DBBS Human and Statistical Genetics Program Retreat. St. Louis, MO. September 2015.
	\item ``Identity-by-descent analysis of sequence data.'' International Genetic Epidemiology Society. Stevenson, WA. October 2012.
\end{itemize}

\hrulefill

\bigskip

\textsc{{\Large Funding}}

\bigskip

Rutgers Open and Affordable Textbooks Award (\$1,000) \hfill 2022-2023 \\
Co-PIs: Steven Foltz, PhD, and Nathan Fried, PhD \\
``Reimagining Statistics in Biology (SBR) with Open and Affordable Materials''

\bigskip

ASCB Promoting Active Learning and Mentoring (PALM) Fellowship (\$2,000) \hfill 2021-2022 \\
Mentor: Nathan Fried, PhD

\bigskip

ALSF Childhood Cancer Data Lab Postdoctoral Training Grant (\$135,925) \hfill 2021-2023 \\
PI: Steven Foltz, PhD

\bigskip

NIH K12 IRACDA Postdoctoral Fellow (PennPORT) (K12 GM081259) \hfill 2020-2023 \\
PI: Janis Burkhardt, PhD

\hrulefill

\bigskip

\textsc{\Large Professional Development}

\bigskip

\textbf{Genomics Education Partnership Training Workshop} \hfill January 4-6, 2023 \\

\textbf{Fundamentals of POGIL Virtual Workshop} \hfill May 16, 2022 \\
\emph{Process Oriented Guided Inquiry Learning} \\

\textbf{The Teaching Center at Washington University in St. Louis} \hfill Fall 2014 -- January 2020 \\
\emph{Professional Development in Teaching Program}
\begin{itemize}
	\item Preparation in Pedagogy (PiP)
		\subitem Foundational and advanced level teaching workshops
		\subitem Qualifying teaching experience
		\subitem Teaching Philosophy Statement
	\item WU-CIRTL ``Practitioner'' level
		\subitem Completed PiP and Scholarship of Teaching and Learning course
\end{itemize}

\textbf{Center for the Integration of Research, Teaching, and Learning (CIRTL)} \\
\emph{National CIRTL Forum}, Drexel University, Philadelphia, PA \hfill October 13-15, 2019 \\

\textbf{Center for the Improvement of Mentored Experience in Research (CIMER)} \\
\emph{Mentorship Training Program}, WashU, St. Louis, MO \hfill April 1-2, 2019 \\

\textbf{Science Education Partnership and Assessment Laboratory (SEPAL @SFSU)} \\
\emph{Evidence-based Teaching for Researchers}, WashU, St. Louis, MO \hfill September 21, 2018 \\

\textbf{Midstates Consortium for Mathematics and Science} \hfill July 6-8, 2018 \\
\emph{Early Career Workshop}, Gustavus Adolphus College,  St. Peter, MN \\

\hrulefill

\bigskip

\textsc{\Large Honors and Awards}

\begin{itemize}
	\item ``Best of \emph{Cell Reports} 2018'' -- co-first author of a frequently cited publication (Driver Fusions and Their Implications in the Development and Treatment of Human Cancers).
	\item ``Outstanding Mentor'' -- recognizing one mentor working with high schoolers through the Washington University Young Scientist Program (25th Anniversary Gala, October 22, 2016).
	\item ``Outstanding Senior in Mathematics'' -- awarded to the top graduating senior in the department (Presbyterian College Honors Convocation, April 15, 2010).
\end{itemize}

\hrulefill

\bigskip

\textsc{\Large Service and Outreach}

\bigskip

\textbf{Virtual Biology Day, Rutgers University -- Camden, Camden, NJ } \\
\emph{Scientific poster judge} \hfill Spring 2021, Fall 2021, Spring 2022

\bigskip

\textbf{Young Scientist Program, Washington University in St. Louis, St. Louis, MO} \\
\emph{Mentoring program cohort leader} \hfill Fall 2014 -- Spring 2018
\begin{itemize}
	\item Led 12 graduate students paired with high schoolers for four years
	\item Coordinated bi-weekly school visits and educational field trips
	\item Facilitated lesson planning and resources available for hands on science activities
\end{itemize}

\textbf{St. Louis Science Center, St. Louis, MO} \hfill Summer 2015 \\
\emph{Exhibit interpreter}
\begin{itemize}
	\item Smithsonian and NHGRI-sponsored exhibit Genome: Unlocking Life's Code
	\item Discussed the scientific, ethical, and societal importance of genomics with museum patrons
\end{itemize}

\hrulefill

\bigskip

\textsc{{\Large Other Work Experience}} \\

\textbf{Across, Nairobi, Kenya} \hfill August 2010 -- July 2011\\
\emph{Institutional and Organizational Development Volunteer} \\
Promoted the organization and its mission (holistic transformation of South Sudan communities) to an international audience through the website and other publications. Key aspects of the organization's mission included: serving refugees and internally displaced people, promoting gender equity and sustainable agriculture, and training teachers and health care providers. Placement coordinated through the Presbyterian Church (U.S.A.) Young Adult Volunteers program. \\

Last updated: \today

\end{document}
