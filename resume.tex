\documentclass[11pt]{article}
\usepackage[letterpaper, margin=1in]{geometry}
\pagestyle{empty}
\usepackage{etaremune}
\frenchspacing
\setlength\parindent{0pt}

\begin{document}

\textsc{{\huge Steven Foltz, PhD}}

\bigskip

Staff Scientist II, Children's Hospital of Philadelphia \hfill stevenmasonfoltz@gmail.com \\
Laboratory of Kai Tan, PhD \hfill https://github.com/envest

\begin{center}
\emph{cancer bioinformatics expert with 10+ years experience | trained educator | positive team leader}
\end{center}

\hrulefill

\bigskip

\textsc{{\Large Research Fluency and Skills}}
\begin{itemize}
\item Bioinformatics pipeline and tool development for cancer multiomics and clinical data analysis
\item Data science, open software development, and visualization best practices in R and python
\item Unix scripting and project management in high-performance / cloud computing environments
\item Next-gen sequencing data processing using bulk DNA and RNA-seq and single-cell multiomics
\item Machine learning algorithm development for biomarker discovery and label prediction
\item Biostatistics analyses and regression, including statistical genetics and survival analysis
\item Cross-disciplinary and multi-institutional collaboration with clinical and wet lab scientists
\item Classroom and boardroom communication of technical material for wide-ranging audiences

\end{itemize}

\hrulefill

\bigskip

\textsc{{\Large Education and Work Experience}}

\bigskip

\textbf{Staff Scientist II, Children's Hospital of Philadelphia} \hfill Since May 2023 \\
Laboratory of Kai Tan, PhD \\
\emph{Focus:} Single-cell multiomics analysis of pediatric tumor evolution and minimal residual disease \\

\textbf{Postdoctoral Fellow, University of Pennsylvania} \hfill July 2020 -- May 2023 \\
Laboratory of Casey Greene, PhD, and the Childhood Cancer Data Lab at ALSF \\
\emph{Focus:} Cross-platform bioinformatics data integration for machine learning in pediatric oncology \\

\textbf{PhD, Washington University in St. Louis} \hfill June 2014 -- March 2020 \\
Division of Biology and Biomedical Sciences, Human and Statistical Genetics Program \\
Laboratory of Li Ding, PhD

\begin{itemize}
  \item Led pan-cancer computational projects utilizing tumor and normal tissue sequencing data to understand patterns of germline and somatic variation
  \item Developed integrative data analysis tools and pipelines, including variant calling and quality control using bulk DNA and RNA, single-cell RNA, and linked-read DNA sequencing
  \item Discovered gene fusions from The Cancer Genome Atlas (9,624 patients, 33 cancer types) and a large multiple myeloma cohort (742 patients), highlighting clinically-relevant events
\end{itemize}

\textbf{Biostatistician, Virginia Commonwealth University} \hfill August 2013 -- May 2014 \\
\textbf{MS Biostatistics, University of Washington} \hfill September 2011 -- August 2013 \\
\textbf{BS Mathematics, Presbyterian College} \hfill August 2006 -- May 2010

\end{document}
